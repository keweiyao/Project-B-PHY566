\documentclass[a4paper,12pt]{article}
\usepackage{enumerate,amsmath,graphicx,float,epstopdf,subcaption}
\graphicspath{ {./} }
\usepackage{verbatim}

\begin{document}

\title{PHY566 Group Project 2 \\ 
	Eco-system: Predator and Prey}
\date{\today}
\author{David Hicks\\ Weiyao Ke \\ Shagun Maheshwari \\ Fan Zhang}

\maketitle
\section{Abstract}

The following is the link to the GitHub account of the work done in this paper: \\
\begin{center}
https://github.com/keweiyao/Project-B-PHY566.git
\end{center}

\section{Introduction to eco-system modelling}

\section{Model description and implementation}
To simulate the interaction between the predator and pray we keep the essential features of the L-V equations:
\begin{enumerate}
	\item Procreation of deer and wolves when reproduction age is reached.
	\item Death of deer and wolves when starvation age is reached.
	\item Wolves feed on deer.
\end{enumerate}

We set up an animal class which generates instance of animals (deer and wolves), an animal takes record on its own age variables (reproducing age variable and starving age variable), its current and previous location. Animals can wander around an $N \times N$ grid world with periodic boundary condition (equivalent to the surface of a torus).

Initially, populations of deer and wolves with certain age structure is generated with each of their members occupied a randomly chosen but mutually exclusive position on the 2D gird. All the animal of the same species have the same age of sexual maturity and starvation.  A $N \times N$ occupation matrix tracks the position of all the animals, with vacant, deer, and wolves labeled by $0, 1$  and $2$ respectively. 

Within each step, the following operations are done serially: 
\begin{enumerate}
\item Step1: reproducing and starving age variable of each member of wolf and deer population are increased by $1$; then we check whether the starving age variable reaches the starvation age of this animal. Only if the animal is still alive shall we keep it in the animal list for the operations below.
\item Step2: loop over all wolf members and generate a list of its neighbours and take down its present location on the grid as previous location. If there is one or more deer around, then the wolf chooses one deer randomly and "eat" it by take over its spatial location and reset its own age back to zero. If there is no deer around but vacant position, the wolf move to one randomly chosen neighbouring position. The new position is then stored and compared with its previous location. If they are not equal and the wolves is mature, it deposits a new-born wolf at the vacant place left behind and reset its reproducing age variable to 0.
\item Step3: loop over all deer and check if its location is currently occupied by a wolf which means the deer is captured! Only those deer not captured will be kept in the animal list for the operations below.
\item Step4: loop over all deer left and move it to a randomly chosen vacant neighbouring position. If the new position is not equal to its previous location and the deer is mature, it deposits a new-born deer at the vacant position left behind and reset its own reproducing age to 0.
\item Step5: go back to Step1 and increase the system time by $1$
\end{enumerate}




\section{Results and discussion}

\end{document}
